% Options for packages loaded elsewhere
\PassOptionsToPackage{unicode}{hyperref}
\PassOptionsToPackage{hyphens}{url}
%
\documentclass[
]{docs}
\usepackage{amsmath,amssymb}
\usepackage{lmodern}
\usepackage{iftex}
\ifPDFTeX
  \usepackage[T1]{fontenc}
  \usepackage[utf8]{inputenc}
  \usepackage{textcomp} % provide euro and other symbols
\else % if luatex or xetex
  \usepackage{unicode-math}
  \defaultfontfeatures{Scale=MatchLowercase}
  \defaultfontfeatures[\rmfamily]{Ligatures=TeX,Scale=1}
\fi
% Use upquote if available, for straight quotes in verbatim environments
\IfFileExists{upquote.sty}{\usepackage{upquote}}{}
\IfFileExists{microtype.sty}{% use microtype if available
  \usepackage[]{microtype}
  \UseMicrotypeSet[protrusion]{basicmath} % disable protrusion for tt fonts
}{}
\makeatletter
\@ifundefined{KOMAClassName}{% if non-KOMA class
  \IfFileExists{parskip.sty}{%
    \usepackage{parskip}
  }{% else
    \setlength{\parindent}{0pt}
    \setlength{\parskip}{6pt plus 2pt minus 1pt}}
}{% if KOMA class
  \KOMAoptions{parskip=half}}
\makeatother
\usepackage{xcolor}
\usepackage{color}
\usepackage{fancyvrb}
\newcommand{\VerbBar}{|}
\newcommand{\VERB}{\Verb[commandchars=\\\{\}]}
\DefineVerbatimEnvironment{Highlighting}{Verbatim}{commandchars=\\\{\}}
% Add ',fontsize=\small' for more characters per line
\usepackage{framed}
\definecolor{shadecolor}{RGB}{248,248,248}
\newenvironment{Shaded}{\begin{snugshade}}{\end{snugshade}}
\newcommand{\AlertTok}[1]{\textcolor[rgb]{0.94,0.16,0.16}{#1}}
\newcommand{\AnnotationTok}[1]{\textcolor[rgb]{0.56,0.35,0.01}{\textbf{\textit{#1}}}}
\newcommand{\AttributeTok}[1]{\textcolor[rgb]{0.77,0.63,0.00}{#1}}
\newcommand{\BaseNTok}[1]{\textcolor[rgb]{0.00,0.00,0.81}{#1}}
\newcommand{\BuiltInTok}[1]{#1}
\newcommand{\CharTok}[1]{\textcolor[rgb]{0.31,0.60,0.02}{#1}}
\newcommand{\CommentTok}[1]{\textcolor[rgb]{0.56,0.35,0.01}{\textit{#1}}}
\newcommand{\CommentVarTok}[1]{\textcolor[rgb]{0.56,0.35,0.01}{\textbf{\textit{#1}}}}
\newcommand{\ConstantTok}[1]{\textcolor[rgb]{0.00,0.00,0.00}{#1}}
\newcommand{\ControlFlowTok}[1]{\textcolor[rgb]{0.13,0.29,0.53}{\textbf{#1}}}
\newcommand{\DataTypeTok}[1]{\textcolor[rgb]{0.13,0.29,0.53}{#1}}
\newcommand{\DecValTok}[1]{\textcolor[rgb]{0.00,0.00,0.81}{#1}}
\newcommand{\DocumentationTok}[1]{\textcolor[rgb]{0.56,0.35,0.01}{\textbf{\textit{#1}}}}
\newcommand{\ErrorTok}[1]{\textcolor[rgb]{0.64,0.00,0.00}{\textbf{#1}}}
\newcommand{\ExtensionTok}[1]{#1}
\newcommand{\FloatTok}[1]{\textcolor[rgb]{0.00,0.00,0.81}{#1}}
\newcommand{\FunctionTok}[1]{\textcolor[rgb]{0.00,0.00,0.00}{#1}}
\newcommand{\ImportTok}[1]{#1}
\newcommand{\InformationTok}[1]{\textcolor[rgb]{0.56,0.35,0.01}{\textbf{\textit{#1}}}}
\newcommand{\KeywordTok}[1]{\textcolor[rgb]{0.13,0.29,0.53}{\textbf{#1}}}
\newcommand{\NormalTok}[1]{#1}
\newcommand{\OperatorTok}[1]{\textcolor[rgb]{0.81,0.36,0.00}{\textbf{#1}}}
\newcommand{\OtherTok}[1]{\textcolor[rgb]{0.56,0.35,0.01}{#1}}
\newcommand{\PreprocessorTok}[1]{\textcolor[rgb]{0.56,0.35,0.01}{\textit{#1}}}
\newcommand{\RegionMarkerTok}[1]{#1}
\newcommand{\SpecialCharTok}[1]{\textcolor[rgb]{0.00,0.00,0.00}{#1}}
\newcommand{\SpecialStringTok}[1]{\textcolor[rgb]{0.31,0.60,0.02}{#1}}
\newcommand{\StringTok}[1]{\textcolor[rgb]{0.31,0.60,0.02}{#1}}
\newcommand{\VariableTok}[1]{\textcolor[rgb]{0.00,0.00,0.00}{#1}}
\newcommand{\VerbatimStringTok}[1]{\textcolor[rgb]{0.31,0.60,0.02}{#1}}
\newcommand{\WarningTok}[1]{\textcolor[rgb]{0.56,0.35,0.01}{\textbf{\textit{#1}}}}
\usepackage{longtable,booktabs,array}
\usepackage{calc} % for calculating minipage widths
% Correct order of tables after \paragraph or \subparagraph
\usepackage{etoolbox}
\makeatletter
\patchcmd\longtable{\par}{\if@noskipsec\mbox{}\fi\par}{}{}
\makeatother
% Allow footnotes in longtable head/foot
\IfFileExists{footnotehyper.sty}{\usepackage{footnotehyper}}{\usepackage{footnote}}
\makesavenoteenv{longtable}
\usepackage{graphicx}
\makeatletter
\def\maxwidth{\ifdim\Gin@nat@width>\linewidth\linewidth\else\Gin@nat@width\fi}
\def\maxheight{\ifdim\Gin@nat@height>\textheight\textheight\else\Gin@nat@height\fi}
\makeatother
% Scale images if necessary, so that they will not overflow the page
% margins by default, and it is still possible to overwrite the defaults
% using explicit options in \includegraphics[width, height, ...]{}
\setkeys{Gin}{width=\maxwidth,height=\maxheight,keepaspectratio}
% Set default figure placement to htbp
\makeatletter
\def\fps@figure{htbp}
\makeatother
\setlength{\emergencystretch}{3em} % prevent overfull lines
\providecommand{\tightlist}{%
  \setlength{\itemsep}{0pt}\setlength{\parskip}{0pt}}
\setcounter{secnumdepth}{5}
\usepackage{booktabs}
\usepackage{amsthm}
\makeatletter
\def\thm@space@setup{%
  \thm@preskip=8pt plus 2pt minus 4pt
  \thm@postskip=\thm@preskip
}
\makeatother
\ifLuaTeX
  \usepackage{selnolig}  % disable illegal ligatures
\fi
\IfFileExists{bookmark.sty}{\usepackage{bookmark}}{\usepackage{hyperref}}
\IfFileExists{xurl.sty}{\usepackage{xurl}}{} % add URL line breaks if available
\urlstyle{same} % disable monospaced font for URLs
\hypersetup{
  pdftitle={Modul Pelatihan JNE},
  pdfauthor={Penulis: Bakti Siregar, S.Si.,M.Sc},
  hidelinks,
  pdfcreator={LaTeX via pandoc}}

\title{Modul Pelatihan JNE}
\usepackage{etoolbox}
\makeatletter
\providecommand{\subtitle}[1]{% add subtitle to \maketitle
  \apptocmd{\@title}{\par {\large #1 \par}}{}{}
}
\makeatother
\subtitle{Analisis Data dengan Menggunakan Python (Pemula)}
\author{Penulis: Bakti Siregar, S.Si.,M.Sc}
\date{2022-09-21}

\begin{document}
\maketitle

{
\setcounter{tocdepth}{2}
\tableofcontents
}
\hypertarget{overview}{%
\section*{Overview}\label{overview}}
\addcontentsline{toc}{section}{Overview}

Peran ilmu analisis data dalam era digital dan big data seperti sekarang ini sangatlah penting karena begitu melimpahnya data yang kita produksi setiap hari, baik itu dari akun media social, ecommerce, media, transportasi/ekpedisi online, youtube, game, perusahaan, dll. Data tersebut menyediakan informasi yang dapat menentukan keputusan penting dalam berbagai sektor industri. Contohnya, perusahaan pengiriman barang seperti di JNE dapat melakukan analisis data dari hal yang paling sederhana hingga kompleks;

\begin{itemize}
\tightlist
\item
  Jumlah transaksi yang dilakukan konsumen setiap hari, minggu, bulan, dan tahunan
\item
  Berapa banyak pengiriman tepat waktu, cacat (hilang), dan pengiriman terlambat
\item
  Rata-rata waktu yang dibutuhkan untuk melakukan pengiriman barang dari suatu daerah ke berbagai daerah
\item
  Mengetahui daerah asal dan tujuan pengiriman terbanyak di seluruh Indonesia
\item
  Pertumbuhan pengguna dari waktu ke waktu
\item
  Jenis produk yang paling sering dikirimkan
\item
  Mitra yang paling sering melakukan transaksi pengiriman barang
\item
  Tren pertumbuhan customer, cost, dan profit perusahaan secara umum maupun parsial
\item
  Analisis tingkat kepuasan pengguna
\item
  Faktor yang mempengaruhi perpindahan pengguna ke competitor
\item
  Dll.
\end{itemize}

Seluruh transaksi pengguna tersebut adalah data yang perlu dipelajari dan dianalisis. Kemudian, dapat diambil keputusan maupun kebijakan bisnis yang dinilai lebih menguntungkan bagi perusahaan seperti melakukan promo, penentuan harga, optimasi operasional pengiriman, dan lain-lain. Semakin akurat analisis data, semakin baik keputusan yang diambil, maka semakin besar profit perusahaan. Banyaknya data dan faktor yang harus dipertimbangkan dalam menganalisis data tidak mudah untuk dilakukan dengan menggunakan alat bantu klasisk seperti Excel. Tren saat ini, analis data biasanya dilakukan dengan bantuan teknologi pemrograman computer seperti R, Python, dan SQL karena dapat digunakan untuk data yang besar dan lebih fleksibel.

\hypertarget{pengenalan-python}{%
\section*{Pengenalan Python?}\label{pengenalan-python}}
\addcontentsline{toc}{section}{Pengenalan Python?}

Python dalam pembelajaran ini bukanlah seokor binatang melata menyeramkan seperti yang kita banyangkan atau seekor ular besar yang pernah kita tonton di televisi.Tetapi, Python yang akan kita gunakan adalah sebuah bahasa pemrograman yang digunakan untuk membuat aplikasi, perintah komputer, dan melakukan analisis data.

\hypertarget{sejarah}{%
\subsection{Sejarah}\label{sejarah}}

\textbf{Guido van Rossum} adalah seorang progammer yang pertama kali mendesain dan mengimplementasikan Python pada bulan Desember 1989. Projek ini mulai dikembangkan di Centrum Wiskunde \& Informatice yang berpusat di Belanda. Awalnya, Guido bermaksud untuk memperbaiki sintak-sintak bermasalah pada Bahasa Pemrograman ABC yang pernah dikerjakan sebelumnya. Sehingga dalam perjalanannya bahasa pemrograman Python banyak dipengaruhi oleh bahasa pemrograman ABC. Menariknya, tidak seperti bahasa pemrograman lainnya yang banyak dikerjakan dan dirilis oleh perusahaan besar dengan melibatkan para profesional. Python justru dikembangkan secara berkesinambungan oleh ribuan Programmer, penguji, dan pengguna yang kebanyakan bukan ahli IT dari seluruh dunia hingga akhirnya menjadi seperti sekarang.

Inspirasi nama Python diambil dari acara TV BBC yang berjudul ``Monty Python's Flying Circus'', karena Guido adalah penggemar berat acara tersebut. Selain itu, ia ingin memberikan nama yang pendek, unik, dan sedikit misterius untuk penemuannya.

Version yang paling banyak digunakan adalah versi dari Python \textbf{2.x} dan \textbf{3.x}, dan masing-masing versi memiliki fanbase dalam penggunannya. Namun, sejak awal Januari 2020 untuk support maintenance pada Python versi \textbf{2.x} telah berhenti. Bahasa pemrograman Python digunakan untuk berbagai tujuan antaranya Developing, Scripting, Generation and Software testing. Karena bahasanya yang elegan dan sederhana, beberapa top perusahaan teknologi seperti Dropbox, Google, Quora, Mozilla, IBM dan Cisco menggunakan Python dalam implementasinya.

\hypertarget{peranan}{%
\subsection{Peranan}\label{peranan}}

Peranan Python sebagai bahasa pemrograman yang populer seperti sekarang ini sudah tidak diragukan lagi. Python dapat membantumu mengembangkan banyak hal, seperti mengembangkan backend, perangkat lunak dan kecerdasan buatan lainnya. Mari kita simak alasan-alasan apa saja yang menguatkanmu untuk belajar Python dikutip dari teknologi.id:

\begin{itemize}
\tightlist
\item
  \textbf{Open source}, dan it is free.
\item
  Python \textbf{mudah digunakan} bahkan untuk seorang pemula tanpa latar belakang IT.
\item
  Python bahasanya \textbf{ekspresif}, karena dapat menyederhanakan code, misalkan satu baris code dapat mengerjakan banyak hal ketimbang satu baris code pada bahasa pemrograman lainnya.
\item
  Python \textbf{mudah dibaca} (dalam hal debug, maintain, dan modify code) oleh manusia, tidak hanya komputer.
\item
  Memiliki \textbf{banyak Library} di dalamnya.
\item
  Dapat \textbf{diterapkan lintas platform} seperti Windows, Mac, Linux, UNIX, dll.
\item
  \textbf{Bahasa pemorograman terpopuler} dari tahun 2018 hingga saat ini berdasarakan analisis Google trend.
\end{itemize}

Keunggulan Python yang bersifat interpretatif juga banyak digunakan untuk prototyping, scripting dalam pengelolaan infrastruktur, hingga pembuatan website berskala besar.Python adalah salah satu bahasa pemrograman yang dianggap sebagai opsi paling tepat untuk seluruh proyek Data Analitik, Sains Data, Artificial Intelligence (AI), termasuk machine learning.

\hypertarget{keunggulan}{%
\subsection{Keunggulan}\label{keunggulan}}

Jika dibandingkan dengan bahasa pemrograman populer lainnya, R adalah salah satu pemrograman yang dianggap menjadi sanding terkuat dalam hal olah data. Olehkarena itu mari kita perhatikan keunggulan dari keduanya sebagai berikut:

\hypertarget{instalasi}{%
\subsection{Instalasi}\label{instalasi}}

Python memiliki banyak versi, dan sangat mungkin jika di dalam Local Machine kita (komputer) memiliki beberapa versi yang berbeda. Kunjungi \href{https://www.anaconda.com/products/distribution}{Anaconda.com} dan download Installer sesuai dengan Operating System (OS) di komputer anda.

\hypertarget{windows}{%
\subsubsection{Windows}\label{windows}}

Berikut ini adalah video tutorial untuk melakukan installasi Anconda di Windows.

\hypertarget{macintosh}{%
\subsubsection{Macintosh}\label{macintosh}}

Berikut ini adalah video tutorial untuk melakukan installasi Anconda di Macintosh (Mac OS).

\hypertarget{linux}{%
\subsubsection{Linux}\label{linux}}

Berikut ini adalah video tutorial untuk melakukan installasi Anconda di Linux.

\hypertarget{penggunaan}{%
\subsection{Penggunaan}\label{penggunaan}}

Sebenarnya ada banyak tools yang dapat digunakan untuk menulis Python script, yaitu:

\begin{itemize}
\tightlist
\item
  \href{https://www.python.org/shell/}{Python Shell},
\item
  \href{https://www.spyder-ide.org/}{Spyder},
\item
  \href{https://www.sublimetext.com/}{Sublime Text},
\item
  \href{https://visualstudio.microsoft.com/vs/features/python/}{Visual Studio},
\item
  \href{https://jupyter.org/}{Jupyterlab (Notebook)},
\item
  \href{https://colab.research.google.com/?hl=id}{Google Colab}, dll.
\end{itemize}

Salah satu yang paling populer yaitu menggunakan Jupyterlab, tool ini sudah pasti akan terinstall otomatis apabila dilakukan proses pemasangan Python menggunakan Anaconda. Jika pemasangan Python tidak melalui Anaconda, untuk menggunakan Jupyterlab maka terlebih dahulu melakukan pemasangan dengan `pip' atau `conda' (Script : \texttt{pip\ install\ jupyterlab}). Salah satu keunggulan Jupyterlab adalah kita bisa melihat integrasi antara input dan output dari script yang kita tulis, di dalam satu tampilan dokumen yang interaktif. Berikut ini adalah tampilan Jupeterlab, biasanya link \url{http://localhost:8888/lab/}:

Selain Jupyterlab, cara alternatif yang paling banyak digunakan untuk menulis koding Python adalah Google Colab:

\textbf{Catatan (Opsional):} Lakukan update sebelum menggunakan Python, istall library paling umum digunakan yaitu; (panda, numpy, matplotlib, seaborn) dengan menggunakan pip dan conda (Contoh: \texttt{pip\ install\ panda} atau \texttt{conda\ install\ panda})

\hypertarget{konsep-dasar-python}{%
\section{Konsep Dasar Python}\label{konsep-dasar-python}}

Untuk memulai mempelajari segala sesuatu perlu dilakukan dari hal yang sangat fundamental, begitu pula jika anda ingin belajar Python. Konsep dasar dan cara kerja Python, dapat dianalogikan sebagai proses memasak Roti serata bagaimana cara menyajiannya. Perhatikan gambar berikut:

Sejatinya koding yang dituliskan dalam bahasa pemrograman Python adalah suatu instruksi dari pengguna \textbf{(user)} kepada komputer untuk melakukan proses mengolah antara bahan, persiapan sampai dengan memasak dibutuhkan ``kata'' khusus, ``simbol'', sehingga dikombinasikan sebagai sebuah ``kalimat''.

\hypertarget{atribut-bahan-bahan}{%
\subsection{Atribut (Bahan-bahan)}\label{atribut-bahan-bahan}}

Setiap bahasa pemrograman memiliki atribut dan aturan penulisan yang berbeda-beda. Python memiliki beberapa aturan penulisan objek, variabel, statement, operator, hingga penulisan komentar. Jika terdapat penulisan yang tidak sesuai dengan aturan, maka program tidak akan berjalan atau error. Berikut ini adalah beberapa atribut dan struktur penulisan yang perlu diperhatikan pengguna Python.

\hypertarget{objek}{%
\subsubsection{Objek}\label{objek}}

Apapun yang diimplementasikan di Python disebut sebagai sebuah objek. Object ini tentu akan memiliki beberapa tipe, misalnya nilai Integer atau Boolean. Berdasarkan tipe suatu objek tersebut, dapat menentukan proses apa yang bisa dilakukan dengan data yang dimiliki. Contoh; suatu nilai Integer dianggap sebagai objek yang dianalogikan dengan sebuah kotak transparan yang memiliki nilai 10.

\textbf{Catatan:} Tipe object, menentukan apakah value yang ada di dalam kotak ini dapat diubah (mutable) atau konstan (immutable).

\hypertarget{variabel}{%
\subsubsection{Variabel}\label{variabel}}

Variabel merupakan tempat penyimpanan sementara yang dapat digunakan untuk menyimpan data atau informasi. Variabel bersifat mutable, artinya nilai yang ada di dalam variabel dapat diubah, Nilai yang ada di dalam variabel pun dapat berupa bilangan maupun kata. Jadi, apabila suatu saat data tersebut ingin ditampilkan, kita hanya perlu memanggil variabel tersebut untuk menampilkan data yang tersimpan. Dengan kata lain variabel dapat dianggap seperti sebuah sticky note yang menempel di sebuah objek.

Penulisan variabel dalam Python juga memiliki aturan tertentu, yaitu:

\begin{itemize}
\tightlist
\item
  Karakter pertama harus berupa huruf atau garis bawah/underscore (\_).
\item
  Karakter selanjutnya dapat berupa huruf, garis bawah/underscore (\_) atau angka.
\item
  Karakter pada nama variabel bersifat sensitif (case-sensitif). Artinya huruf kecil dan huruf besar dibedakan. Sebagai contoh, variabel nama dan Nama dianggap menjadi variabel yang berbeda.
\end{itemize}

\url{https://medium.com/analytics-vidhya/data-types-in-python-506009234f89}

\begin{Shaded}
\begin{Highlighting}[]
\NormalTok{x }\OperatorTok{=} \DecValTok{2}
\NormalTok{y }\OperatorTok{=} \DecValTok{3} 
\NormalTok{z }\OperatorTok{=} \DecValTok{5} 
\BuiltInTok{print}\NormalTok{(x,y,z)}
\end{Highlighting}
\end{Shaded}

\begin{verbatim}
## 2 3 5
\end{verbatim}

Selain menyimpan atau pengisian nilai, ada juga menjumlahkan, mengurangi, perkalian, pembagian, dsb.

\begin{longtable}[]{@{}cc@{}}
\toprule()
\endhead
\textbf{Operator} & \textbf{Simbol} \\
Penjumlahan & += \\
Pengurangan & -= \\
Perkalian & *= \\
Pembagian & /= \\
Sisa Bagi & \%= \\
Pemangkatan **= & \\
\bottomrule()
\end{longtable}

Perhatikan contoh berikut:

\begin{Shaded}
\begin{Highlighting}[]
\NormalTok{x }\OperatorTok{+=} \DecValTok{2}
\BuiltInTok{print}\NormalTok{(x)}
\end{Highlighting}
\end{Shaded}

\begin{verbatim}
## 4
\end{verbatim}

\hypertarget{komentar}{%
\subsubsection{Komentar}\label{komentar}}

Menambahkan/Memberikan komentar dalam skrip/koding Python adalah untuk memudahkan anda memahami arti/makna penggunaan suatu perintah/program. Komentar yang ditulis dalam sebuah program tersebut hanya bersifat penjelasan tentang apa yang dilakukannya atau apa yang seharusnya dilakukan oleh sebuah skrip/koding. Perlu dicatat bahwa komentar yang bersifat informasi tidak ada hubungannya dengan logika pemrogaram yang sedang anda gunakan. Mereka benar-benar diabaikan oleh kompiler dan dengan demikian tidak pernah tercermin pada input. Biasanya komentar dituliskan pada satu baris yang tersedia di Python, dengan menggunakan \texttt{\#} di awal maupun di akhir pernyataan.

\begin{Shaded}
\begin{Highlighting}[]
\CommentTok{\# Mengganti nilai x yang sudah direkam (Komentar di awal pernyataan) }
\NormalTok{x}\OperatorTok{=}\DecValTok{12} 
\NormalTok{z }\OperatorTok{=}\NormalTok{ x }\OperatorTok{+}\NormalTok{ y   }\CommentTok{\# Mengganti nilai z yang sudah direkam (Komentar di akhir pernyataan) }
\end{Highlighting}
\end{Shaded}

\hypertarget{operator}{%
\subsection{Operator}\label{operator}}

Operator adalah simbol yang mengarahkan compiler untuk melakukan berbagai macam operasi terhadap beberapa penugasan. Operator mensimulasikan berbagai operasi matematis, logika, dan keputusan yang dilakukan pada sekumpulan Bilangan Kompleks, Integer, dan Numerik sebagai penugasan masukan (input). R dan Python mendukung sebagian besar empat jenis operator biner antara satu set penugasan. Dalam ini, kita akan melihat berbagai jenis operator yang tersedia di R dan Python dan penggunaannya.

\hypertarget{aritmatika}{%
\subsubsection{Aritmatika}\label{aritmatika}}

Penggunaan operator aritmatika dalam program R dan Python adalah untuk mensimulasikan berbagai operasi matematika, seperti penambahan, pengurangan, perkalian, pembagian, dan modulo. Operator aritmatika yang dilakukan bisa saja berupa nilai skalar, bilangan kompleks, atau vektor.

\begin{longtable}[]{@{}cc@{}}
\toprule()
\endhead
\textbf{Operator} & \textbf{Python} \\
Penjumlahan & + \\
Pengurangan & - \\
Perkalian & * \\
Divisi/Pembagian & / \\
Pemangkatan & ** \\
Modulo & \% \\
\bottomrule()
\end{longtable}

Untuk pemahaman lebih lanjut, perhatikan cuplikan Python berikut:

\textbf{Catatan:} Terlebih dahulu install \texttt{numpy} di Anaconda Navgator anda dengan cara membuka lingkungan \emph{(environment)} \texttt{py38} yang sudah anda buatkan sebelumnya.

\begin{Shaded}
\begin{Highlighting}[]
\ImportTok{import}\NormalTok{ numpy }\ImportTok{as}\NormalTok{ np      }\CommentTok{\# fungsi untuk operasi pada Array}
\NormalTok{x }\OperatorTok{=}\NormalTok{ np.array([}\DecValTok{2}\NormalTok{,}\DecValTok{3}\NormalTok{,}\DecValTok{5}\NormalTok{])   }\CommentTok{\# memuat vektor x}
\NormalTok{y }\OperatorTok{=}\NormalTok{ np.array([}\DecValTok{2}\NormalTok{,}\DecValTok{4}\NormalTok{,}\DecValTok{6}\NormalTok{])   }\CommentTok{\# memuat vektor y}
\NormalTok{x}\OperatorTok{+}\NormalTok{y                     }\CommentTok{\# hasil penjumahan vektor x dan y}
\end{Highlighting}
\end{Shaded}

\begin{verbatim}
## array([ 4,  7, 11])
\end{verbatim}

\begin{Shaded}
\begin{Highlighting}[]
\BuiltInTok{print}\NormalTok{ (x}\OperatorTok{+}\NormalTok{y)             }\CommentTok{\# hasil penjumahan vektor x dan y}
\end{Highlighting}
\end{Shaded}

\begin{verbatim}
## [ 4  7 11]
\end{verbatim}

\begin{Shaded}
\begin{Highlighting}[]
\BuiltInTok{print}\NormalTok{ (x}\OperatorTok{{-}}\NormalTok{y)             }\CommentTok{\# hasil pengurangan vektor x dan y}
\end{Highlighting}
\end{Shaded}

\begin{verbatim}
## [ 0 -1 -1]
\end{verbatim}

\begin{Shaded}
\begin{Highlighting}[]
\BuiltInTok{print}\NormalTok{ (x}\OperatorTok{*}\NormalTok{y)             }\CommentTok{\# hasil perkalian vektor x dan y}
\end{Highlighting}
\end{Shaded}

\begin{verbatim}
## [ 4 12 30]
\end{verbatim}

\begin{Shaded}
\begin{Highlighting}[]
\BuiltInTok{print}\NormalTok{ (x}\OperatorTok{/}\NormalTok{y)             }\CommentTok{\# hasil pembagian vektor x dan y}
\end{Highlighting}
\end{Shaded}

\begin{verbatim}
## [1.         0.75       0.83333333]
\end{verbatim}

\begin{Shaded}
\begin{Highlighting}[]
\BuiltInTok{print}\NormalTok{ (x}\OperatorTok{**}\NormalTok{y)            }\CommentTok{\# hasil pemangkatan vektor x dan y}
\end{Highlighting}
\end{Shaded}

\begin{verbatim}
## [    4    81 15625]
\end{verbatim}

\begin{Shaded}
\begin{Highlighting}[]
\BuiltInTok{print}\NormalTok{ (x}\OperatorTok{\%}\NormalTok{y)             }\CommentTok{\# hasil modulo (Sisa bagi) vektor x dan y}
\end{Highlighting}
\end{Shaded}

\begin{verbatim}
## [0 3 5]
\end{verbatim}

Adakalanya anda perlu menampilkan keterangan/komentar yang juga melekat pada hasil perhitungan Python itu sendiri. Maka anda dapat melakukannya dengan cara berikut:

\begin{Shaded}
\begin{Highlighting}[]
\BuiltInTok{print}\NormalTok{(}\StringTok{"Penjumahan vektor x dan y :"}\NormalTok{, x }\OperatorTok{+}\NormalTok{ y, }\StringTok{"}\CharTok{\textbackslash{}n}\StringTok{"}\NormalTok{)}
\end{Highlighting}
\end{Shaded}

\begin{verbatim}
## Penjumahan vektor x dan y : [ 4  7 11]
\end{verbatim}

\begin{Shaded}
\begin{Highlighting}[]
\BuiltInTok{print}\NormalTok{(}\StringTok{"Pengurangan vektor x dan y :"}\NormalTok{, x }\OperatorTok{{-}}\NormalTok{ y, }\StringTok{"}\CharTok{\textbackslash{}n}\StringTok{"}\NormalTok{)}
\end{Highlighting}
\end{Shaded}

\begin{verbatim}
## Pengurangan vektor x dan y : [ 0 -1 -1]
\end{verbatim}

\begin{Shaded}
\begin{Highlighting}[]
\BuiltInTok{print}\NormalTok{(}\StringTok{"Perkalian vektor x dan y :"}\NormalTok{, x }\OperatorTok{*}\NormalTok{ y, }\StringTok{"}\CharTok{\textbackslash{}n}\StringTok{"}\NormalTok{)}
\end{Highlighting}
\end{Shaded}

\begin{verbatim}
## Perkalian vektor x dan y : [ 4 12 30]
\end{verbatim}

\begin{Shaded}
\begin{Highlighting}[]
\BuiltInTok{print}\NormalTok{(}\StringTok{"Pembagian vektor x dan y :"}\NormalTok{, x }\OperatorTok{/}\NormalTok{ y, }\StringTok{"}\CharTok{\textbackslash{}n}\StringTok{"}\NormalTok{)}
\end{Highlighting}
\end{Shaded}

\begin{verbatim}
## Pembagian vektor x dan y : [1.         0.75       0.83333333]
\end{verbatim}

\begin{Shaded}
\begin{Highlighting}[]
\BuiltInTok{print}\NormalTok{(}\StringTok{"Pemangkatan vektor x dan y :"}\NormalTok{, x }\OperatorTok{**}\NormalTok{ y) }
\end{Highlighting}
\end{Shaded}

\begin{verbatim}
## Pemangkatan vektor x dan y : [    4    81 15625]
\end{verbatim}

\begin{Shaded}
\begin{Highlighting}[]
\BuiltInTok{print}\NormalTok{(}\StringTok{"Modulo vektor x dan y :"}\NormalTok{, x }\OperatorTok{\%}\NormalTok{ y, }\StringTok{"}\CharTok{\textbackslash{}n}\StringTok{"}\NormalTok{)}
\end{Highlighting}
\end{Shaded}

\begin{verbatim}
## Modulo vektor x dan y : [0 3 5]
\end{verbatim}

\textbf{Catatan:} Penjelasan lebih lekap mengenai modulo dapat lihat pada \href{https://www.omnicalculator.com/math/modulo}{link ini}

\hypertarget{relasional}{%
\subsubsection{Relasional}\label{relasional}}

Operator relasional melakukan operasi perbandingan antara elemen yang bersesuaian pada setiap operan. Mengembalikan nilai Boolean TRUE jika operan pertama memenuhi relasi dibandingkan dengan operan kedua. Nilai TRUE selalu dianggap lebih besar dari FALSE.

\begin{longtable}[]{@{}
  >{\centering\arraybackslash}p{(\columnwidth - 4\tabcolsep) * \real{0.3077}}
  >{\centering\arraybackslash}p{(\columnwidth - 4\tabcolsep) * \real{0.3077}}
  >{\centering\arraybackslash}p{(\columnwidth - 4\tabcolsep) * \real{0.3846}}@{}}
\toprule()
\endhead
\textbf{Operator} & \textbf{Python} & \textbf{Keterangan} \\
Kurang dari & \textless{} & Mengembalikan TRUE jika elemen yang bersesuaian pada operan pertama lebih kecil dari operan kedua. Selain itu akan mengembalikan FALSE \\
Kurang dari sama dengan & \textless= & Mengembalikan TRUE jika elemen yang bersesuaian pada operan pertama kurang dari atau sama dengan elemen operan kedua. Selain itu akan mengembalikan FALSE \\
Lebih besar dari & \textgreater{} & Mengembalikan TRUE jika elemen yang bersesuaian pada operan pertama lebih besar dari operan kedua. Selain itu akan mengembalikan FALSE \\
Lebih besar dari sama dengan & \textgreater= & Mengembalikan BENAR jika elemen yang bersesuaian pada operan pertama lebih besar atau sama dengan dari operan kedua. Selain itu akan mengembalikan FALSE \\
Sama Dengan & == & Mengembalikan BENAR jika dan hanya jika kedua sisi bernilai sama \\
Tidak Sama dengan & != & Mengembalikan BENAR jika elemen yang bersesuaian pada operan pertama tidak sama dengan dari operan kedua \\
\bottomrule()
\end{longtable}

\begin{Shaded}
\begin{Highlighting}[]
\ImportTok{import}\NormalTok{ numpy }\ImportTok{as}\NormalTok{ np    }\CommentTok{\# fungsi untuk operasi pada Array}
\NormalTok{x }\OperatorTok{=}\NormalTok{ np.array([}\DecValTok{2}\NormalTok{,}\DecValTok{3}\NormalTok{,}\DecValTok{5}\NormalTok{]) }\CommentTok{\# memuat vektor x}
\NormalTok{y }\OperatorTok{=}\NormalTok{ np.array([}\DecValTok{2}\NormalTok{,}\DecValTok{4}\NormalTok{,}\DecValTok{6}\NormalTok{]) }\CommentTok{\# memuat vektor y}
\BuiltInTok{print}\NormalTok{(}\StringTok{"Vektor x  kurang dari Vektor y:"}\NormalTok{, x }\OperatorTok{\textless{}}\NormalTok{ y, }\StringTok{"}\CharTok{\textbackslash{}n}\StringTok{"}\NormalTok{)}
\end{Highlighting}
\end{Shaded}

\begin{verbatim}
## Vektor x  kurang dari Vektor y: [False  True  True]
\end{verbatim}

\begin{Shaded}
\begin{Highlighting}[]
\BuiltInTok{print}\NormalTok{(}\StringTok{"Vector x kurang dari sama dengan Vector y:"}\NormalTok{, x }\OperatorTok{\textless{}=}\NormalTok{ y, }\StringTok{"}\CharTok{\textbackslash{}n}\StringTok{"}\NormalTok{)}
\end{Highlighting}
\end{Shaded}

\begin{verbatim}
## Vector x kurang dari sama dengan Vector y: [ True  True  True]
\end{verbatim}

\begin{Shaded}
\begin{Highlighting}[]
\BuiltInTok{print}\NormalTok{(}\StringTok{"Vector x lebih besar dari Vector y :"}\NormalTok{, x }\OperatorTok{\textgreater{}}\NormalTok{ y, }\StringTok{"}\CharTok{\textbackslash{}n}\StringTok{"}\NormalTok{)}
\end{Highlighting}
\end{Shaded}

\begin{verbatim}
## Vector x lebih besar dari Vector y : [False False False]
\end{verbatim}

\begin{Shaded}
\begin{Highlighting}[]
\BuiltInTok{print}\NormalTok{(}\StringTok{"Vector x lebih besar dari sama dengan Vector y :"}\NormalTok{, x }\OperatorTok{\textgreater{}=}\NormalTok{ y, }\StringTok{"}\CharTok{\textbackslash{}n}\StringTok{"}\NormalTok{)}
\end{Highlighting}
\end{Shaded}

\begin{verbatim}
## Vector x lebih besar dari sama dengan Vector y : [ True False False]
\end{verbatim}

\begin{Shaded}
\begin{Highlighting}[]
\BuiltInTok{print}\NormalTok{(}\StringTok{"Vector x sama dengan Vector y:"}\NormalTok{, x }\OperatorTok{==}\NormalTok{ y,}\StringTok{"}\CharTok{\textbackslash{}n}\StringTok{"}\NormalTok{) }
\end{Highlighting}
\end{Shaded}

\begin{verbatim}
## Vector x sama dengan Vector y: [ True False False]
\end{verbatim}

\begin{Shaded}
\begin{Highlighting}[]
\BuiltInTok{print}\NormalTok{(}\StringTok{"Vector x tidak sama dengan Vector y:"}\NormalTok{, x }\OperatorTok{!=}\NormalTok{ y) }
\end{Highlighting}
\end{Shaded}

\begin{verbatim}
## Vector x tidak sama dengan Vector y: [False  True  True]
\end{verbatim}

\hypertarget{logika}{%
\subsubsection{Logika}\label{logika}}

Operator logis mensimulasikan operasi keputusan, berdasarkan operator yang ditentukan antara operan, yang kemudian dievaluasi ke nilai Boolean Benar atau Salah. Nilai bilangan bulat bukan nol dianggap sebagai nilai BENAR, baik itu bilangan kompleks atau bilangan real.

\begin{longtable}[]{@{}
  >{\centering\arraybackslash}p{(\columnwidth - 4\tabcolsep) * \real{0.3077}}
  >{\centering\arraybackslash}p{(\columnwidth - 4\tabcolsep) * \real{0.3077}}
  >{\centering\arraybackslash}p{(\columnwidth - 4\tabcolsep) * \real{0.3846}}@{}}
\toprule()
\endhead
\textbf{Operator} & \textbf{Python} & \textbf{Keterangan} \\
NOT & ! & Operasi negasi/kebalikan pada status elemen operan \\
AND & \& & Mengembalikan TRUE jika kedua operan bernilai Benar \\
OR & \(|\) & Mengembalikan TRUE jika salah satu operan adalah Benar \\
XOR & \^{} & Mengembalikan TRUE jika salah satu dari kedua elemen pertama operan bernilai Benar \\
\bottomrule()
\end{longtable}

\begin{Shaded}
\begin{Highlighting}[]
\ImportTok{import}\NormalTok{ numpy }\ImportTok{as}\NormalTok{ np                }\CommentTok{\# fungsi untuk operasi pada Array}
\NormalTok{x }\OperatorTok{=}\NormalTok{ np.array([}\VariableTok{False}\NormalTok{,}\VariableTok{True}\NormalTok{,}\VariableTok{False}\NormalTok{])  }\CommentTok{\# memuat vektor x}
\NormalTok{y }\OperatorTok{=}\NormalTok{ np.array([}\VariableTok{True}\NormalTok{,}\VariableTok{True}\NormalTok{,}\VariableTok{False}\NormalTok{])   }\CommentTok{\# memuat vektor y}
 
\CommentTok{\# Melakukan operasi logika pada Operan}
\BuiltInTok{print}\NormalTok{(}\StringTok{"Logika Negasi (\textasciitilde{}) untuk vektor x:"}\NormalTok{, }\OperatorTok{\textasciitilde{}}\NormalTok{x, }\StringTok{"}\CharTok{\textbackslash{}n}\StringTok{"}\NormalTok{) }
\end{Highlighting}
\end{Shaded}

\begin{verbatim}
## Logika Negasi (~) untuk vektor x: [ True False  True]
\end{verbatim}

\begin{Shaded}
\begin{Highlighting}[]
\BuiltInTok{print}\NormalTok{(}\StringTok{"Logika Negasi (\textasciitilde{}) untuk vektor y :"}\NormalTok{, }\OperatorTok{\textasciitilde{}}\NormalTok{y, }\StringTok{"}\CharTok{\textbackslash{}n}\StringTok{"}\NormalTok{) }
\end{Highlighting}
\end{Shaded}

\begin{verbatim}
## Logika Negasi (~) untuk vektor y : [False False  True]
\end{verbatim}

\begin{Shaded}
\begin{Highlighting}[]
\BuiltInTok{print}\NormalTok{(}\StringTok{"Logika Konjungsi (Dan) :"}\NormalTok{, x }\OperatorTok{\&}\NormalTok{ y, }\StringTok{"}\CharTok{\textbackslash{}n}\StringTok{"}\NormalTok{)}
\end{Highlighting}
\end{Shaded}

\begin{verbatim}
## Logika Konjungsi (Dan) : [False  True False]
\end{verbatim}

\begin{Shaded}
\begin{Highlighting}[]
\BuiltInTok{print}\NormalTok{(}\StringTok{"Logika Disjungsi (Atau) :"}\NormalTok{, x }\OperatorTok{|}\NormalTok{ y, }\StringTok{"}\CharTok{\textbackslash{}n}\StringTok{"}\NormalTok{)}
\end{Highlighting}
\end{Shaded}

\begin{verbatim}
## Logika Disjungsi (Atau) : [ True  True False]
\end{verbatim}

\begin{Shaded}
\begin{Highlighting}[]
\BuiltInTok{print}\NormalTok{(}\StringTok{"Logika Disjungsi Parsial :"}\NormalTok{, x }\OperatorTok{\^{}}\NormalTok{ y)}
\end{Highlighting}
\end{Shaded}

\begin{verbatim}
## Logika Disjungsi Parsial : [ True False False]
\end{verbatim}

\textbf{Catatan:} Dalam Python, hanya menggunakan True dan False bukan dalam hurup Kapital.

\hypertarget{lain-lain}{%
\subsubsection{Lain-lain}\label{lain-lain}}

Berikut ini juga ada beberapa operator yang kemungkinan besar juga akan anda perlukan pada saat akan menggunakan Python.

\begin{Shaded}
\begin{Highlighting}[]
\ImportTok{import}\NormalTok{ numpy }\ImportTok{as}\NormalTok{ np            }\CommentTok{\# library untuk memuat vektor (array)}
\NormalTok{x }\OperatorTok{=} \DecValTok{2}                         \CommentTok{\# memuat vektor x}
\NormalTok{y }\OperatorTok{=} \DecValTok{3}                         \CommentTok{\# memuat vektor y}
\ImportTok{from}\NormalTok{ math }\ImportTok{import}\NormalTok{ log,exp,sqrt }\CommentTok{\# mengambil beberapa fungsi dari library \textasciigrave{}math\textasciigrave{}}
\NormalTok{sqrt(x}\OperatorTok{*}\NormalTok{y)                     }\CommentTok{\# Bentuk akar}
\end{Highlighting}
\end{Shaded}

\begin{verbatim}
## 2.449489742783178
\end{verbatim}

\begin{Shaded}
\begin{Highlighting}[]
\NormalTok{log(x)                        }\CommentTok{\# logaritma }
\end{Highlighting}
\end{Shaded}

\begin{verbatim}
## 0.6931471805599453
\end{verbatim}

\begin{Shaded}
\begin{Highlighting}[]
\NormalTok{exp(y)                        }\CommentTok{\# eksponen}
\end{Highlighting}
\end{Shaded}

\begin{verbatim}
## 20.085536923187668
\end{verbatim}

\begin{Shaded}
\begin{Highlighting}[]
\NormalTok{(x}\OperatorTok{/}\NormalTok{y) }\OperatorTok{+}\NormalTok{ y                     }\CommentTok{\# Tanda kurung}
\end{Highlighting}
\end{Shaded}

\begin{verbatim}
## 3.6666666666666665
\end{verbatim}

\begin{Shaded}
\begin{Highlighting}[]
\ImportTok{import}\NormalTok{ numpy }\ImportTok{as}\NormalTok{ np      }\CommentTok{\# library untuk memuat vektor (array)}
\NormalTok{x }\OperatorTok{=}\NormalTok{ np.array([}\DecValTok{2}\NormalTok{,}\DecValTok{3}\NormalTok{,}\DecValTok{5}\NormalTok{])   }\CommentTok{\# memuat vektor x}
\NormalTok{y }\OperatorTok{=}\NormalTok{ np.array([}\DecValTok{2}\NormalTok{,}\DecValTok{4}\NormalTok{,}\DecValTok{6}\NormalTok{])   }\CommentTok{\# memuat vektor y}
\NormalTok{np.sqrt(x}\OperatorTok{*}\NormalTok{y)            }\CommentTok{\# Bentuk akar}
\end{Highlighting}
\end{Shaded}

\begin{verbatim}
## array([2.        , 3.46410162, 5.47722558])
\end{verbatim}

\begin{Shaded}
\begin{Highlighting}[]
\NormalTok{np.log(x)               }\CommentTok{\# logaritma }
\end{Highlighting}
\end{Shaded}

\begin{verbatim}
## array([0.69314718, 1.09861229, 1.60943791])
\end{verbatim}

\begin{Shaded}
\begin{Highlighting}[]
\NormalTok{np.exp(y)               }\CommentTok{\# eksponen}
\end{Highlighting}
\end{Shaded}

\begin{verbatim}
## array([  7.3890561 ,  54.59815003, 403.42879349])
\end{verbatim}

\begin{Shaded}
\begin{Highlighting}[]
\NormalTok{(x}\OperatorTok{/}\NormalTok{y) }\OperatorTok{+}\NormalTok{ y               }\CommentTok{\# Tanda kurung}
\end{Highlighting}
\end{Shaded}

\begin{verbatim}
## array([3.        , 4.75      , 6.83333333])
\end{verbatim}

\textbf{Catatan:} \href{https://www.advernesia.com/blog/matematika/sifat-komutatif-asosiatif-dan-distributif/}{Sifat Komutatif Asosiatif dan Distributif} juga berlaku dalam program Python.

\hypertarget{tipe-data}{%
\subsection{Tipe Data}\label{tipe-data}}

Dalam pemrograman seperti Python, tipe data merupakan konsep penting. Keduanya dapat menggunakan variabel untuk menyimpan tipe yang berbeda-beda, berikut adalah tipe data paling mendasar yang harus diketahui:

\begin{longtable}[]{@{}
  >{\centering\arraybackslash}p{(\columnwidth - 4\tabcolsep) * \real{0.4286}}
  >{\centering\arraybackslash}p{(\columnwidth - 4\tabcolsep) * \real{0.2381}}
  >{\centering\arraybackslash}p{(\columnwidth - 4\tabcolsep) * \real{0.3333}}@{}}
\toprule()
\endhead
\textbf{Tipe Data} & \textbf{Python} & \textbf{Penjelasan} \\
Double/Float & 5.6 & Bilangan yang mempunyai koma \\
Integer & 5 & Bilangan bulat 1,2,\ldots,n \\
Bolean/Logical & True/False & Benar bernilai 1 dan Salah bernilai 0 \\
String/Character & `Dsciencelabs' & karakter/kalimat bisa berupa huruf angka, dll (diapit tanda '' atau ') \\
Complex & 1 + 5j & Pasangan angka real dan imajiner \\
\bottomrule()
\end{longtable}

Berikut ini adalah koding Python yang dapat digunakan untuk menetapkan kelima tipe data diatas:

\begin{Shaded}
\begin{Highlighting}[]
\NormalTok{d1 }\OperatorTok{=} \FloatTok{5.6}             \CommentTok{\# Tetapkan nilai desimal }
\NormalTok{d2 }\OperatorTok{=} \DecValTok{5}               \CommentTok{\# tetapkan nilai integer}
\NormalTok{d3 }\OperatorTok{=}\NormalTok{ [}\VariableTok{True}\NormalTok{,}\VariableTok{False}\NormalTok{]    }\CommentTok{\# list Bolean/Logical}
\NormalTok{d4 }\OperatorTok{=}\NormalTok{ [}\StringTok{"a"}\NormalTok{,}\StringTok{\textquotesingle{}b\textquotesingle{}}\NormalTok{,}\StringTok{\textquotesingle{}123\textquotesingle{}}\NormalTok{] }\CommentTok{\# list String/Character}
\NormalTok{d5 }\OperatorTok{=} \DecValTok{1} \OperatorTok{+} \OtherTok{5j}          \CommentTok{\# Complex }
\end{Highlighting}
\end{Shaded}

Untuk memeriksa tipe data dalam Python:

\begin{Shaded}
\begin{Highlighting}[]
\BuiltInTok{type}\NormalTok{(d5)             }\CommentTok{\# cetak tipe data}
\end{Highlighting}
\end{Shaded}

\begin{verbatim}
## <class 'complex'>
\end{verbatim}

\hypertarget{bantuan}{%
\subsection{Bantuan}\label{bantuan}}

Salah satu bagian terpenting dalam bekerja dengan bahasa Python adalah mengetahui di mana mencari bantuan.

\begin{Shaded}
\begin{Highlighting}[]
\BuiltInTok{help}\NormalTok{(}\BuiltInTok{print}\NormalTok{)}
\end{Highlighting}
\end{Shaded}

\begin{verbatim}
## Help on built-in function print in module builtins:
## 
## print(...)
##     print(value, ..., sep=' ', end='\n', file=sys.stdout, flush=False)
##     
##     Prints the values to a stream, or to sys.stdout by default.
##     Optional keyword arguments:
##     file:  a file-like object (stream); defaults to the current sys.stdout.
##     sep:   string inserted between values, default a space.
##     end:   string appended after the last value, default a newline.
##     flush: whether to forcibly flush the stream.
\end{verbatim}

\hypertarget{pemrograman-python}{%
\section{Pemrograman Python}\label{pemrograman-python}}

Here is a review of existing methods.

\hypertarget{persiapan-data}{%
\section{Persiapan Data}\label{persiapan-data}}

We describe our methods in this chapter.

\url{https://www.dqlab.id/kenali-4-fungsi-penting-pandas-python-untuk-pengolahan-data}

\hypertarget{visualisasi-data}{%
\section{Visualisasi Data}\label{visualisasi-data}}

Some \emph{significant} applications are demonstrated in this chapter.

\hypertarget{example-one}{%
\subsection{Example one}\label{example-one}}

\hypertarget{example-two}{%
\subsection{Example two}\label{example-two}}

\hypertarget{studi-kasus}{%
\section{Studi Kasus}\label{studi-kasus}}

We have finished a nice book.

\hypertarget{pertemuan-1}{%
\subsection{Pertemuan 1}\label{pertemuan-1}}

\begin{enumerate}
\def\labelenumi{\arabic{enumi}.}
\tightlist
\item
  \url{https://www.tutorialspoint.com/r/index.htm}
\item
  \url{https://www.statista.com/chart/16567/popular-programming-languages/}
\item
  \url{https://www.anotherbookondatascience.com/chapter1.html}
\item
  \url{https://www.kodefungsi.com/fungsi/PYTHON/}
\item
  \url{https://www.anotherbookondatascience.com/chapter1.html}
\item
  \url{https://www.petanikode.com}
\item
  \url{https://stiki-indonesia.ac.id/2021/03/23/perkembangan-python-dan-alasan-sangat-penting-untuk-data-analitik/}
\end{enumerate}

\end{document}
